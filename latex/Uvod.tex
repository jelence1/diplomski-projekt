\chapter{Uvod}

Internet stvari (engl. Internet of things - IoT) je brzo rastuća tehnologija koja urbani svijet pretvara u potpuno mrežno povezani sustav visoke tehnologije. Uz sve veću popularnost, potražnju i rastuće zahtjeve na IoT tehologiju, sustavi generiraju sve veću količinu podataka koju je gotovo nemoguće obraditi lokalno. Tom je izazovu doskočila tehnologija računarstva u oblaku (engl. \textit{cloud computing}), koja pruža softver, alate i infrastrukturu preko interneta umjesto lokalnog upravljanja. Računanje u oblaku može pružiti skalabilnost i dostupnost potrebnu za pohranu i obradu velike količine podataka koju stvaraju IoT sustavi. Osim toga, računanje u oblaku može pomoći u smanjenju troškova upravljanja IoT sustavima. 

Za razvoj IoT uređaja potrebni su mikrokontroleri s mogućnošću bežičnog povezivanja, primarno Wi-Fi komunikacije. Serija ESP32 mikrokontrolera tvrtke \textit{Espressif} ozbiljan je konkurent među bežičnim uređajima zbog niske potrošnje, visoke otpornosti na temperature, te najvažnije, jednostavnom bežičnom povezivosti \cite{top_25_iot}. Jedan takav čip je ESP32-C3, koji pruža Wi-Fi i Bluetooth povezivanje. Čip je integriran u nekoliko različitih modula, koji su pak dio razvojnih sustava koje proizvodi \textit{Espressif}. Za izradu ovog rada odabran je razvojni sustav ESP32-C3-DevKitM-1. Isto tako, usluga računarstva u oblaku mora biti jednostavna, intuitivna, pouzdana te najvažnije, skalabilna. Platforma AWS tvrtke \textit{Amazon} ističe se kao jedan od najkorištenijih sustava za računarstvo u oblaku.

Ovaj projekt analizira mogućnosti koje pruža ESP32-C3-DevKitM-1 u razvoju aplikacija koje koriste Wi-Fi te kako povezati modul s oblakom platforme AWS. Opisan je i sam modul te mogućnosti koje nudi u sklopu Wi-Fi povezivanja. Dan je osnovni pregled funkcionalnosti koje nudi AWS te su nabrojane njegove usluge za olakšano stvaranje IoT sustava. Isto tako, opisan je postupak povezivanja modula i oblaka koji nudi AWS. Također, uz pomoć perifernih uređaja stvoren je pokazni IoT sustav koji integrira modul i računalni oblak.

Rad je podijeljen u cjeline kako slijedi. Drugo poglavlje „\textit{Razvojni sustav ESP32- C3-DevKitM-1}“ opisuje osnovne karakteristike korištenog razvojnog sustava kao ciljane hardverske platforme te su opisane najvažnije značajke W-Fi tehnologije. U trećem poglavlju „\textit{Amazon Web Services (AWS)}“ dan je pregled ekosustava AWS u kontekstu IoT sustava te su navedene usluge koje AWS podržava za olakšanu integraciju fizičkih uređaja s oblakom. Četvrto poglavlje „\textit{Integracija računalnog oblaka i razvojnog sustava}“ opisuje postupke povezivanja razvojnog sustava s oblakom AWS usluge te je modeliran manji IoT sustav koji koristi opisane tehnologije uz dodatnu periferiju radi simulacije stvarnog sustava. Za kraj su opisana razmatranja o mogućim proširenjima ovakvog postava. 

\eject