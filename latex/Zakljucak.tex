\chapter{Zaključak}

Detaljno je opisan postupak integracije razvojnog sustava ESP32-C3-DevKitM-1 s računalnim oblakom platforme AWS. Također su opisana tehnička svojstva razvojnog sustava te je detaljnije opisana Wi-Fi tehnologija. Dan je pregled ključnih funkcionalnosti koje pruža oblak u sustavu AWS za IoT sustave, te su ukratko opisane dodatne značajke koje se mogu koristiti u skalabilnim IoT sustavima. Opisana su programska rješenja za spajanje modula i usluge AWS koje je potrebno implementirati u razvojnom sustavu. Također, navedena su moguća poboljšanja postojećih rješenja za buduće integracije.

Integracijom opisanih tehnologija dobiva se kompletan IoT uređaj koji se može koristiti u svakodnevnom životu. Korištene demo implementacije mogu se skalirati na više uređaja unutar platforme AWS. Kombinacija ESP32-C3-DevKitM-1 razvojnog sustava, Wi-Fi veze i integracije s platformom AWS pruža snažno i fleksibilno rješenje za razvoj IoT aplikacija. Ovaj integrirani sustav omogućuje prikupljanje, pohranu i upravljanje
podacima iz različitih senzora ili uređaja te njihovo daljnje korištenje unutar platforme AWS, gdje se pak razne usluge računarstva u oblaku mogu koristiti za obradu podataka. Zahvaljujući ovakvoj konfiguraciji, korisnici mogu stvoriti napredne IoT sustave s visokom razinom kontrole, praćenja i automatizacije.

\eject